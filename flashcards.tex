\documentclass[17pt]{extarticle}
\usepackage[paperheight=2.75in,paperwidth=4.75in,scale=.9]{geometry}
\usepackage{amssymb} % lettered bullet points
\usepackage{amsmath}
\usepackage{_shortcuts}

\usepackage{fontspec}
\usepackage{unicode-math}

\setmainfont{TeX Gyre Bonum}
\setmathfont{TeX Gyre Bonum Math}

\usepackage{pgfmorepages}
\pgfmorepagesloadextralayouts
\pgfpagesuselayout{5 index cards}[a4paper]

\title{FM To Learn}
\author{ksh }
\date{March 2023}

%% Makes it easier to maintain and to change the style of all of them in one go
\NewDocumentCommand\IndexCard { +m +m }
{
  %% Vertically and horizontally centre the "name" side 
  \vspace*{\fill}%
  \begin{center}
  #1%
  \end{center}
  \vspace*{\fill}%
  \newpage
  %% Vertically centre the explanation side
  \vspace*{\fill}%
  #2
  \vspace*{\fill}%
  \newpage
}

\let\vec=\mathbf

\begin{document}
\pagestyle{empty}
\IndexCard{%%
Vector equation of a line with cross product
}{%%
\[
    (\vec{r}-\vec{a})\times \vec{b}=\vec{0}
\]
where
\begin{itemize}
\item \(\vec{a}\) is the position vector of a point the line passes through
\item \(\vec{b}\) is the direction vector
\end{itemize}
}

\IndexCard{%%
Shortest distance between skew lines
\begin{align*}
l_1 &: \vec{r}=\vec{a}+\lambda \vec{b} \\
l_2 &: \vec{r}=\vec{c}+\mu \vec{d}
\end{align*}
}{%
  \[
    \left|
    \frac{
      (\vec{a}-\vec{c}).(\vec{b}\times \vec{d})
    }{
      |\vec{b}\times\vec{d}|
    }
    \right|
    \]
}

\IndexCard{%
Area of a Triangle
    }{%
\[
    \frac{1}{2}|\vec{a}\times \vec{b}|
    \]
}

\IndexCard{%
Area of a Parallelogram
    }{%
\[
    |\vec{a}\times \vec{b}|
    \]
}

\IndexCard{%
Volume of a Parallelepiped
    }{%
\[
    |\vec{a} \cdot \vec{b}\times \vec{c}|
    \]
}

\IndexCard{%
Volume of a Tetrahedron
    }{%
\[
\frac{1}{6}|\vec{a} \cdot \vec{b}\times \vec{c}|
    \]
}

\IndexCard{%
\(t\)--formula for \(\sin \theta\)\\
where \(t=\tan \frac{\theta}{2}\)
    }{%
\[
\sin \theta=\frac{2t}{1+t^2}
\]
}


\IndexCard{%
\(t\)--formula for \(\cos \theta\)\\
where \(t=\tan \frac{\theta}{2}\)
    }{%
\[
\cos \theta=\frac{1-t^2}{1+t^2}
\]
}


\IndexCard{%
\(t\)--formula for \(\tan \theta\)\\
where \(t=\tan \frac{\theta}{2}\)
    }{%
\[
\tan \theta=\frac{2t}{1-t^2}
\]
}

\IndexCard{%
\(t\)--formula for \(\dydx{t}{\theta}\)\\
(Weierstrass substitution)\\
where \(t=\tan \frac{\theta}{2}\)
    }{%
\[
\dydx{t}{\theta}=\frac{1}{2}(1+t^2)
\]
}

\IndexCard{%
Simpson's rule for 
\[\int_a^b y\; \mathrm{d}x\]
    }{%
\begin{align*}
\approx \frac{1}{3} h & \big( (y_0+y_{2n})\\
 & +4(y_1+...+y_{2n-1}) \\
 & +2(y_2+...+y_{2n-2}) \big)
\end{align*}
where \(h=\frac{b-a}{2n}\)
}

\IndexCard{%
Taylor series powers of \(x\)
    }{%
\[
f(x+a)=\sum_{n=0}^\infty{\frac{f^{(n)}(a)x^n}{n!}}
\]
}


\IndexCard{%
Leibnitz's formula for 
%
\[
  \dydx{^ny}{x^n}
\]
    }{%
\[
\dydx{^ny}{x^n}= \sum_{k=0}^n{\binom{n}{k} \dydx{^ku}{x^k}\dydx{^{(n-k)}u}{x^{(n-k)}}}
\]
}


\IndexCard{%
\(3 \times 3\) matrix for reflection in the plane \(x=0\)
    }{%
\[
\begin{pmatrix}
-1 &0&0\\0&1&0\\0&0&1
\end{pmatrix}
\]
}


\IndexCard{%
\(3 \times 3\) matrix for reflection in the plane \(y=0\)
    }{%
\[
\begin{pmatrix}
1 &0&0\\0&-1&0\\0&0&1
\end{pmatrix}
\]
}


\IndexCard{%
\(3 \times 3\) matrix for reflection in the plane \(z=0\)
    }{%
\[
\begin{pmatrix}
1 &0&0\\0&1&0\\0&0&-1
\end{pmatrix}
\]
}


\IndexCard{%
\(3 \times 3\) matrix for rotation about the \(x\)--axis
    }{%
\[
\begin{pmatrix}
1 &0&0\\0&\cos \theta&-\sin\theta\\0&\sin\theta&\cos\theta
\end{pmatrix}
\]
}


\IndexCard{%
\(3 \times 3\) matrix for rotation about the \(y\)--axis
    }{%
\[
\begin{pmatrix}
\cos\theta &0&\sin\theta\\0&1&0\\-\sin\theta&0&\cos\theta
\end{pmatrix}
\]
}


\IndexCard{%
\(3 \times 3\) matrix for rotation about the \(z\)--axis
    }{%
\[
\begin{pmatrix}
\cos\theta &-\sin\theta&0\\\sin\theta&\cos\theta&0\\0&0&1
\end{pmatrix}
\]
}



\IndexCard{%
Complex loci \(|z-a|=b\)
    }{%
Circle centre \(a\), radius \(b\)
}


\IndexCard{%
Complex loci \(|z-a|=|z-b|\)
    }{%
Perpendicular bisector of the line joining the complex numbers \(a\) and \(b\)
}


\IndexCard{%
Complex loci \(\operatorname{arg}(z-a)=\beta\)
    }{%
Half line from the complex number \(a\) at an angle of \(\beta\) to the real axis.
}


\IndexCard{%
Direction cosines of the line
%
\[
  \vec{r}=\vec{a}+\lambda\vec{b}
\]
%
 where \(\vec{b}=\begin{pmatrix}x\\y\\z\end{pmatrix}\)
    }{%
{\small
The angle that the line makes with the axes:
\begin{itemize}
\item \(x\)-axis: \(\cos\alpha = \frac{x}{|\vec{b}|}\)\\
\item \(y\)-axis: \(\cos\beta = \frac{y}{|\vec{b}|}\)\\
\item \(z\)-axis: \(\cos\gamma = \frac{z}{|\vec{b}|}\)
\end{itemize}
}
}

\IndexCard{%
Polar graph \(r=p \sec(\alpha-\theta)\)
    }{%
Straight line\\
Convert into \(y=mx+c\) using the addition formula for cosine
}

\IndexCard{%
Polar graph \(r=a\)
    }{%
Circle centre the pole, radius \(a\)
}

\IndexCard{%
Polar graph \(r=2a\cos \theta\)
    }{%
Circle centre \((a,0)\), radius \(a\)\\
One symmetric petal
}

\IndexCard{%
Polar graph \(r=\theta\)
    }{%
Spiral centred at origin
}

\IndexCard{%
Polar graph \(r=a\cos n\theta\)
    }{%
Only considering \(r\geqslant 0\) \\
Rose with \(n\) petals spaced at \(\frac{2\pi}{n}\) starting at \((a,0)\)
}

\IndexCard{%
Polar graph \(r=p+q\cos \theta\)\\
\(q\leqslant p<2q\)
    }{%
Concave curve ``dimple'' shaped limaçon
}

\IndexCard{%
Polar graph \(r=p+q\cos \theta\)\\
 \(p=q\)
    }{%
Cardioid
}

\IndexCard{%
Polar graph \(r=p+q\cos \theta\)\\
\(p\geqslant 2q\)
    }{%
 Convex curve ``egg'' shaped limaçon
}

\IndexCard{%
Polar graph \(r^2=a^2\cos 2\theta\)
    }{%
Leminscate (figure-8 shaped)\\ 
starting at \((a,0)\)
}
\end{document}
