
\documentclass[12pt]{article}
\usepackage{graphicx} % Required for inserting images
\usepackage{xcolor} %colour package
\usepackage{setspace}
\usepackage{enumitem} % numbering package
\usepackage{amssymb} % lettered bullet points
\usepackage{amsmath}
\usepackage{tikz}
\usetikzlibrary{scopes}
\usetikzlibrary{positioning}
\usepackage{tcolorbox}
\tcbuselibrary{skins}
\usepackage{_shortcuts}

\usepackage[explicit]{titlesec}
\title{FM To Learn}
\author{ksh }
\date{March 2023}

\begin{document}
\pagestyle{empty}
\begin{tcolorbox}[arc=0mm,
                skin=bicolor,
                colback=white,
                colbacklower=white,
                sidebyside,
                valign = center,
                overlay={\draw[tcbcolframe, line width=.5mm] (segmentation.north)--(segmentation.south);},
                equal height group=A]

    \vspace{2mm}
Vector equation of a line with cross product
    \tcblower
$$(\textbf{r}-\textbf{a})\times \textbf{b}=\textbf{0}$$\\
where\\
\textbf{a} is the position vector of a point the line passes through\\
\textbf{b} is the direction vector
\end{tcolorbox}
\vspace{2cm}
\begin{tcolorbox}[arc=0mm,
                skin=bicolor,
                sidebyside,
                equal height group=A,
                colback=white,
                colbacklower=white,
                valign = center,
                overlay={\draw[tcbcolframe, line width=.5mm] (segmentation.north)--(segmentation.south);}]

    \vspace{2mm}
Shortest distance between skew lines
\[ l_1: \textbf{r}=\textbf{a}+\lambda \textbf{b}\]
\[ l_2: \textbf{r}=\textbf{c}+\mu \textbf{d}\]
    \tcblower
\[\left|\frac{(\textbf{a}-\textbf{c}).(\textbf{b}\times \textbf{d})}{|\textbf{b}\times\textbf{d}|}\right|\]
 
\end{tcolorbox}
\vspace{2cm}
\begin{tcolorbox}[arc=0mm,
                skin=bicolor,
                sidebyside,
                equal height group=A,
                colback=white,
                colbacklower=white,
                valign = center,
                overlay={\draw[tcbcolframe, line width=.5mm] (segmentation.north)--(segmentation.south);}]

    \vspace{2mm}
Area of a triangle


    \tcblower
\[\frac{1}{2}|\textbf{a}\times \textbf{b}|\]
\end{tcolorbox}
\vspace{2cm}
\begin{tcolorbox}[arc=0mm,
                skin=bicolor,
                sidebyside,
                equal height group=A,
                colback=white,
                colbacklower=white,
                valign = center,
                overlay={\draw[tcbcolframe, line width=.5mm] (segmentation.north)--(segmentation.south);}]

    \vspace{2mm}
Area of a parallelogram


    \tcblower
\[|\textbf{a}\times \textbf{b}|\]
\end{tcolorbox}
\vspace{2cm}
\begin{tcolorbox}[arc=0mm,
                skin=bicolor,
                sidebyside,
                equal height group=A,
                colback=white,
                colbacklower=white,
                valign = center,
                overlay={\draw[tcbcolframe, line width=.5mm] (segmentation.north)--(segmentation.south);}]

    \vspace{2mm}
Volume of a Parallelepiped


    \tcblower
\[|\textbf{a}.\textbf{b}\times \textbf{c}|\]
\end{tcolorbox}
\vspace{2cm}
\begin{tcolorbox}[arc=0mm,
                skin=bicolor,
                sidebyside,
                equal height group=A,
                colback=white,
                colbacklower=white,
                valign = center,
                overlay={\draw[tcbcolframe, line width=.5mm] (segmentation.north)--(segmentation.south);}]

    \vspace{2mm}
Volume of a Tetrahedron


    \tcblower
\[\frac{1}{6}|\textbf{a}.\textbf{b}\times \textbf{c}|\]
\end{tcolorbox}
\vspace{2cm}
\begin{tcolorbox}[arc=0mm,
                skin=bicolor,
                sidebyside,
                equal height group=A,
                colback=white,
                colbacklower=white,
                valign = center,
                overlay={\draw[tcbcolframe, line width=.5mm] (segmentation.north)--(segmentation.south);}]

    \vspace{2mm}
\textit{t}-formula for $\sin \theta$\\
where $t=\tan \frac{\theta}{2}$
    \tcblower
\[\sin \theta=\frac{2t}{1+t^2}\]
\end{tcolorbox}
\vspace{2cm}

\begin{tcolorbox}[arc=0mm,
                skin=bicolor,
                sidebyside,
                equal height group=A,
                colback=white,
                colbacklower=white,
                valign = center,
                overlay={\draw[tcbcolframe, line width=.5mm] (segmentation.north)--(segmentation.south);}]

    \vspace{2mm}
\textit{t}-formula for $\cos \theta$\\
where $t=\tan \frac{\theta}{2}$
    \tcblower
\[\cos \theta=\frac{1-t^2}{1+t^2}\]
\end{tcolorbox}

\vspace{2cm}
\begin{tcolorbox}[arc=0mm,
                skin=bicolor,
                sidebyside,
                equal height group=A,
                colback=white,
                colbacklower=white,
                valign = center,
                overlay={\draw[tcbcolframe, line width=.5mm] (segmentation.north)--(segmentation.south);}]

    \vspace{2mm}
\textit{t}-formula for $\tan \theta$\\
where $t=\tan \frac{\theta}{2}$
    \tcblower
\[\tan \theta=\frac{2t}{1-t^2}\]
\end{tcolorbox}
\vspace{2cm}
\begin{tcolorbox}[arc=0mm,
                skin=bicolor,
                sidebyside,
                equal height group=A,
                colback=white,
                colbacklower=white,
                valign = center,
                overlay={\draw[tcbcolframe, line width=.5mm] (segmentation.north)--(segmentation.south);}]

    \vspace{2mm}
\textit{t}-formula for $\dydx{t}{\theta}$\\
(Weierstrass substitution)\\
where $t=\tan \frac{\theta}{2}$
    \tcblower
\[\dydx{t}{\theta}=\frac{1}{2}(1+t^2)\]
\end{tcolorbox}
\vspace{2cm}
\begin{tcolorbox}[arc=0mm,
                skin=bicolor,
                sidebyside,
                equal height group=A,
                colback=white,
                colbacklower=white,
                valign = center,
                halign lower=center,
                overlay={\draw[tcbcolframe, line width=.5mm] (segmentation.north)--(segmentation.south);}]

    \vspace{2mm}
Simpson's rule for 
\[\int_a^b y\; \mathrm{d}x\]
    \tcblower
\begin{align*}
\approx \frac{1}{3}h & \{ (y_0+y_{2n})\\ & +4(y_1+...+y_{2n-1}) \\ & +2(y_2+...+y_{2n-2})\}
\end{align*}
where $h=\frac{b-a}{2n}$

\end{tcolorbox}
\vspace{2cm}
\begin{tcolorbox}[arc=0mm,
                skin=bicolor,
                sidebyside,
                equal height group=A,
                colback=white,
                colbacklower=white,
                valign = center,
                overlay={\draw[tcbcolframe, line width=.5mm] (segmentation.north)--(segmentation.south);}]

    \vspace{2mm}
Taylor series powers of $x$
    \tcblower
\[f(x+a)=\sum_{n=0}^\infty{\frac{f^{(n)}(a)x^n}{n!}}\]
\end{tcolorbox}
\vspace{2cm}

\begin{tcolorbox}[arc=0mm,
                skin=bicolor,
                sidebyside,
                equal height group=A,
                colback=white,
                colbacklower=white,
                valign = center,
                overlay={\draw[tcbcolframe, line width=.5mm] (segmentation.north)--(segmentation.south);}]

    \vspace{2mm}
Leibnitz's formula for $$\dydx{^ny}{x^n}$$
    \tcblower
\[\dydx{^ny}{x^n}= \sum_{k=0}^n{\binom{n}{k} \dydx{^ku}{x^k}\dydx{^{(n-k)}u}{x^{(n-k)}}}\]
\end{tcolorbox}
\vspace{2cm}

\begin{tcolorbox}[arc=0mm,
                skin=bicolor,
                sidebyside,
                equal height group=A,
                colback=white,
                colbacklower=white,
                valign = center,
                halign lower= center,
                overlay={\draw[tcbcolframe, line width=.5mm] (segmentation.north)--(segmentation.south);}]

    \vspace{2mm}
$3 \times 3$ matrix for reflection in the plane $x=0$
    \tcblower
$\begin{pmatrix}
-1 &0&0\\0&1&0\\0&0&1
\end{pmatrix}$
\end{tcolorbox}
\vspace{2cm}

\begin{tcolorbox}[arc=0mm,
                skin=bicolor,
                sidebyside,
                equal height group=A,
                colback=white,
                colbacklower=white,
                valign = center,
                halign lower= center,
                overlay={\draw[tcbcolframe, line width=.5mm] (segmentation.north)--(segmentation.south);}]

    \vspace{2mm}
$3 \times 3$ matrix for reflection in the plane $y=0$
    \tcblower
$\begin{pmatrix}
1 &0&0\\0&-1&0\\0&0&1
\end{pmatrix}$
\end{tcolorbox}
\vspace{2cm}

\begin{tcolorbox}[arc=0mm,
                skin=bicolor,
                sidebyside,
                equal height group=A,
                colback=white,
                colbacklower=white,
                valign = center,
                halign lower= center,
                overlay={\draw[tcbcolframe, line width=.5mm] (segmentation.north)--(segmentation.south);}]

    \vspace{2mm}
$3 \times 3$ matrix for reflection in the plane $z=0$
    \tcblower
$\begin{pmatrix}
1 &0&0\\0&1&0\\0&0&-1
\end{pmatrix}$
\end{tcolorbox}
\vspace{2cm}

\begin{tcolorbox}[arc=0mm,
                skin=bicolor,
                sidebyside,
                equal height group=A,
                colback=white,
                colbacklower=white,
                valign = center,
                halign lower= center,
                overlay={\draw[tcbcolframe, line width=.5mm] (segmentation.north)--(segmentation.south);}]

    \vspace{2mm}
$3 \times 3$ matrix for rotation about the $x$-axis
    \tcblower
$\begin{pmatrix}
1 &0&0\\0&\cos \theta&-\sin\theta\\0&\sin\theta&\cos\theta
\end{pmatrix}$
\end{tcolorbox}
\vspace{2cm}

\begin{tcolorbox}[arc=0mm,
                skin=bicolor,
                sidebyside,
                equal height group=A,
                colback=white,
                colbacklower=white,
                valign = center,
                halign lower= center,
                overlay={\draw[tcbcolframe, line width=.5mm] (segmentation.north)--(segmentation.south);}]

    \vspace{2mm}
$3 \times 3$ matrix for rotation about the $y$-axis
    \tcblower
$\begin{pmatrix}
\cos\theta &0&\sin\theta\\0&1&0\\-\sin\theta&0&\cos\theta
\end{pmatrix}$
\end{tcolorbox}
\vspace{2cm}

\begin{tcolorbox}[arc=0mm,
                skin=bicolor,
                sidebyside,
                equal height group=A,
                colback=white,
                colbacklower=white,
                valign = center,
                halign lower= center,
                overlay={\draw[tcbcolframe, line width=.5mm] (segmentation.north)--(segmentation.south);}]

    \vspace{2mm}
$3 \times 3$ matrix for rotation about the $z$-axis
    \tcblower
$\begin{pmatrix}
\cos\theta &-\sin\theta&0\\\sin\theta&\cos\theta&0\\0&0&1
\end{pmatrix}$
\end{tcolorbox}

\vspace{2cm}

\begin{tcolorbox}[arc=0mm,
                skin=bicolor,
                sidebyside,
                equal height group=A,
                colback=white,
                colbacklower=white,
                valign = center,
                halign lower= center,
                overlay={\draw[tcbcolframe, line width=.5mm] (segmentation.north)--(segmentation.south);}]

    \vspace{2mm}
Complex loci $|z-a|=b$
    \tcblower
Circle centre $a$, radius $b$
\end{tcolorbox}
\vspace{2cm}

\begin{tcolorbox}[arc=0mm,
                skin=bicolor,
                sidebyside,
                equal height group=A,
                colback=white,
                colbacklower=white,
                valign = center,
                halign lower= center,
                overlay={\draw[tcbcolframe, line width=.5mm] (segmentation.north)--(segmentation.south);}]

    \vspace{2mm}
Complex loci $|z-a|=|z-b|$
    \tcblower
Perpendicular bisector of the line joining the complex numbers $a$ and $b$
\end{tcolorbox}
\vspace{2cm}

\begin{tcolorbox}[arc=0mm,
                skin=bicolor,
                sidebyside,
                equal height group=A,
                colback=white,
                colbacklower=white,
                valign = center,
                halign lower=left,
                overlay={\draw[tcbcolframe, line width=.5mm] (segmentation.north)--(segmentation.south);}]

    \vspace{2mm}
Complex loci $\mathrm{arg}(z-a)=\beta$
    \tcblower
Half line from the complex number $a$ at an angle of $\beta$ to the real axis
\end{tcolorbox}
\vspace{2cm}

\begin{tcolorbox}[arc=0mm,
                skin=bicolor,
                sidebyside,
                equal height group=A,
                colback=white,
                colbacklower=white,
                valign = center,
                overlay={\draw[tcbcolframe, line width=.5mm] (segmentation.north)--(segmentation.south);}]

    \vspace{2mm}
Direction cosines of the line\\ \(\textbf{r}=\textbf{a}+\lambda\textbf{b}\),\\ where \(\textbf{b}=\begin{pmatrix}x\\y\\z\end{pmatrix}\)
    \tcblower
The angle that the line makes with the axes.\\
Angle with the \(x\)-axis, \(\alpha\)\\ \(cos\alpha = \frac{x}{|b|}\)\\
Angle with the \(y\)-axis, \(\beta\)\\ \(cos\beta = \frac{y}{|b|}\)\\
Angle with the \(z\)-axis, \(\gamma\)\\ \(cos\gamma = \frac{z}{|b|}\)
\end{tcolorbox}
\vspace{2cm}
\begin{tcolorbox}[arc=0mm,
                skin=bicolor,
                sidebyside,
                equal height group=A,
                colback=white,
                colbacklower=white,
                valign = center,
                overlay={\draw[tcbcolframe, line width=.5mm] (segmentation.north)--(segmentation.south);}]

    \vspace{2mm}
Polar graph \(r=p \sec(\alpha-\theta)\)
    \tcblower
Straight line\\Convert into \(y=mx+c\) using the addition formula for cosine
\end{tcolorbox}
\vspace{2cm}
\begin{tcolorbox}[arc=0mm,
                skin=bicolor,
                sidebyside,
                equal height group=A,
                colback=white,
                colbacklower=white,
                valign = center,
                overlay={\draw[tcbcolframe, line width=.5mm] (segmentation.north)--(segmentation.south);}]

    \vspace{2mm}
Polar graph \(r=a\)
    \tcblower
Circle centre the pole, radius \(a\)
\end{tcolorbox}
\vspace{2cm}
\begin{tcolorbox}[arc=0mm,
                skin=bicolor,
                sidebyside,
                equal height group=A,
                colback=white,
                colbacklower=white,
                valign = center,
                overlay={\draw[tcbcolframe, line width=.5mm] (segmentation.north)--(segmentation.south);}]

    \vspace{2mm}
Polar graph \(r=2a\cos \theta\)
    \tcblower
Circle centre \((a,0)\), radius \(a\)\\
One symmetric petal
\end{tcolorbox}
\vspace{2cm}
\begin{tcolorbox}[arc=0mm,
                skin=bicolor,
                sidebyside,
                equal height group=A,
                colback=white,
                colbacklower=white,
                valign = center,
                overlay={\draw[tcbcolframe, line width=.5mm] (segmentation.north)--(segmentation.south);}]

    \vspace{2mm}
Polar graph \(r=\theta\)
    \tcblower
Spiral centred at origin
\end{tcolorbox}
\vspace{2cm}
\begin{tcolorbox}[arc=0mm,
                skin=bicolor,
                sidebyside,
                equal height group=A,
                colback=white,
                colbacklower=white,
                valign = center,
                overlay={\draw[tcbcolframe, line width=.5mm] (segmentation.north)--(segmentation.south);}]

    \vspace{2mm}
Polar graph \(r=a\cos n\theta\)
    \tcblower
Only considering \(r\geqslant 0\) \\Rose with \(n\) petals spaced at \(\frac{2\pi}{n}\) starting at \((a,0)\)
\end{tcolorbox}
\vspace{2cm}
\begin{tcolorbox}[arc=0mm,
                skin=bicolor,
                sidebyside,
                equal height group=A,
                colback=white,
                colbacklower=white,
                valign = center,
                halign lower=left,                
                overlay={\draw[tcbcolframe, line width=.5mm] (segmentation.north)--(segmentation.south);}]

    \vspace{2mm}
Polar graph \(r=p+q\cos \theta\)\\ \(q\leqslant p<2q\)
    \tcblower
Concave curve ``dimple" shaped limaçon

\end{tcolorbox}
\vspace{2cm}
\begin{tcolorbox}[arc=0mm,
                skin=bicolor,
                sidebyside,
                equal height group=A,
                colback=white,
                colbacklower=white,
                valign = center,
                halign lower=left,                
                overlay={\draw[tcbcolframe, line width=.5mm] (segmentation.north)--(segmentation.south);}]

    \vspace{2mm}
Polar graph \(r=p+q\cos \theta\)\\ \(p=q\)
    \tcblower
Cardioid\\


\end{tcolorbox}
\vspace{2cm}
\begin{tcolorbox}[arc=0mm,
                skin=bicolor,
                sidebyside,
                equal height group=A,
                colback=white,
                colbacklower=white,
                valign = center,
                halign lower=left,                
                overlay={\draw[tcbcolframe, line width=.5mm] (segmentation.north)--(segmentation.south);}]

    \vspace{2mm}
Polar graph \(r=p+q\cos \theta\)\\\(p\geqslant 2q\)
    \tcblower
 Convex curve ``egg" shaped limaçon
\end{tcolorbox}
\vspace{2cm}
\begin{tcolorbox}[arc=0mm,
                skin=bicolor,
                sidebyside,
                equal height group=A,
                colback=white,
                colbacklower=white,
                valign = center,
                overlay={\draw[tcbcolframe, line width=.5mm] (segmentation.north)--(segmentation.south);}]

    \vspace{2mm}
Polar graph \(r^2=a^2\cos 2\theta\)
    \tcblower
Leminscate (figure-8 shaped)\\ starting at \((a,0)\)
\end{tcolorbox}
\end{document}
